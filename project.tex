\documentclass[12pt, a4paper]{article}

\newcommand{\modulename}{Graph Theory}
\newcommand{\projectyear}{2017}
\newcommand{\projectname}{Project \projectyear}
\newcommand{\duedate}{23/04/2017}

\usepackage{xcolor}
\usepackage{minted}
\usepackage[utf8]{inputenc}
\usepackage{tikz}
\usepackage{caption}
\usepackage{gensymb}
\usepackage{lmodern}
\usepackage{multirow}
\usepackage{booktabs}
\usepackage{array}
\usepackage{adjustbox}
\usepackage{upquote}
\usepackage{amsmath}
\usepackage{url}
\usepackage{fancyhdr}
\usepackage{lastpage}
\usepackage{abstract}

% Abstract
\renewcommand{\abstractname}{}
\renewcommand{\absnamepos}{empty}

% Header and footer
\fancypagestyle{plain}{
  \renewcommand{\headrulewidth}{0pt}
  \lhead{}
  \rhead{}
  \cfoot{Page~\thepage~of~\pageref{LastPage}}
  \renewcommand{\footrulewidth}{0.1pt}
}

\pagestyle{fancy}
\fancyhf{}
\lhead{\modulename}
\rhead{\projectname}
\cfoot{Page~\thepage~of~\pageref{LastPage}}
\renewcommand{\headrulewidth}{0.1pt}
\renewcommand{\footrulewidth}{0.1pt}

% Links
\usepackage[hidelinks]{hyperref}

% Drawing
\usetikzlibrary{mindmap,shadows, shapes, arrows, positioning}
\tikzstyle{rect} = [rectangle, fill=ProcessBlue, text width=4.5em, text centered, minimum height=4em, rounded corners]
\tikzstyle{line} = [draw, ->, very thick]
\tikzstyle{oval} = [ellipse, fill=SeaGreen, text width=5em, text centered]

% Tables
\newcolumntype{x}[1]{>{\centering\arraybackslash\hspace{0pt}}p{#1}}

% Bibliography
\renewcommand{\refname}{\selectfont\normalsize References} 

\title{\projectname}
\author{\modulename}
\date{Due: \duedate}

\begin{document}
\maketitle

\noindent
The following document contains the instructions for \projectname{} for \modulename{}.
The project will be worth 50\% of your mark for this module.
You are required to 

\subsection*{Minimum Viable Project}
The minimum standard for this project is a GitHub repository containing

A better project will be organised, with extensive comments and explanations.


\subsection*{Submissions}
GitHub must be used to manage the development of the software.
Your GitHub repository will form the main submission of the project.
You must submit the URL of your GitHub repository for this project using the link on the course Moodle page before the deadline.
You can do this at any time, as the last commit before the deadline will be used as your submission for this project.

Any submission that does not have a full and incremental git history with informative commit messages over the course of the project work will be accorded a proportionate mark.
It is expected that your repository will have at least tens of commits, with each commit relating to a reasonably small unit of work.
In the last week of term, or at any other time, you may be asked by the lecturer to explain the contents of your git repository.
While it is encouraged that students will engage in peer learning, all software that is contained in your submission must have been written by you yourself.
You can show this by having a long incremental commit history and by being able to explain your code.

\subsection*{Marking scheme}
The following marking scheme will be used to mark the project.
Students should note, however, that in certain circumstances the examiner's overall impression of the project may influence marks in each individual component.

\begin{center}
\begin{tabular}{lll}
\toprule
40\% & \textbf{Programming} & Well-written code; logical file structure.\\
\midrule
40\% & \textbf{Explanation} & Extensive comments; excellent README. \\
\midrule
20\% & \textbf{Management} & Incremental work; informative commit messages. \\
\bottomrule
\end{tabular}
\end{center}

\subsection*{Advice for students}
\begin{itemize}
    \item
    Your git log history should be extensive.
    A reasonable unit of work for a single commit is a small function, or a handful of comments, or a small change that fixes a bug.
    If you are well organised you will find it easier to determine the size of a reasonable commit, and it will show in your git history.
    \item
    Using code from outside sources is sometimes acceptable -- so long as the code is licensed to permit this, you clearly reference the source, and the overall project is substantially your own work.
    Using a source that does not meet these three conditions could jeopardise your mark.
    \item
    You must be able to explain your project during it, and after it.
    Bear this in mind when you are writing your README.
    If you had trouble understanding something in the first place, you will likely have trouble explaining it a couple of weeks later.
    Write a short explanation of it in your README, so that you can jog your memory later.
    \item
    Everyone is susceptible to procrastination and disorganisation.
    You are expected to be aware of this and take reasonable measures to avoid them.
    The best way to do this is to draw up an initial straight-forward project plan and keep it updated.
    You can show the examiner that you have done this in a number of ways.
    The easiest is to summarise the project plan in your README.
    Another way is to use a to-do list like GitHub Issues.
    \item
    Students have problems with projects from time to time.
    Some of these are unavoidable, such as external factors relating to family issues or illness.
    In such cases allowances can sometimes be made.
    Other problems are preventable, such as missing the submission deadline because you are having internet connectivity issues five minutes before.
    Students should be able to show that up until an issue arose they had completed a reasonable and proportionate amount of work, and took reasonable steps to avoid the issue.
    \item
    Go easy on yourself -- this is one project in one module.
    It will not define you or your life.
    A higher overall course mark should not be determined by a single project, but rather your performance in all your work in all your modules.
    Here, you are just trying to demonstrate to yourself, to the examiners, and to prospective future employers, that you can take a reasonably straight-forward problem and solve it within a few weeks.
\end{itemize}


\bibliographystyle{plain}
\bibliography{bibliography}
\end{document}
